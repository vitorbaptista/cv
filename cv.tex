%% start of file `template_en.tex'.
%% Copyright 2007 Xavier Danaux (xdanaux@gmail.com).
%
% This work may be distributed and/or modified under the
% conditions of the LaTeX Project Public License version 1.3c,
% available at http://www.latex-project.org/lppl/.


\documentclass[11pt,a4paper]{moderncv}

% moderncv themes
%\moderncvtheme[blue]{casual}                 % optional argument are 'blue' (default), 'orange', 'red', 'green', 'grey' and 'roman' (for roman fonts, instead of sans serif fonts)
\moderncvtheme[green]{classic}                % idem

% character encoding
\usepackage[utf8]{inputenc}                   % replace by the encoding you are using

% adjust the page margins
\usepackage[scale=0.8]{geometry}
\recomputelengths                             % required when changes are made to page layout lengths

% personal data
\firstname{Vitor}
\familyname{Baptista}
\title{Web Developer}               % optional, remove the line if not wanted
%\address{\hspace{-1in}851 S. Morgan (M/C 249)}{Chicago, IL 60607-7045}    % optional, remove the line if not wanted
%\mobile{+55 (83) 9630-0657}                    % optional, remove the line if not wanted
%\phone{(312) 413-8265}                      % optional, remove the line if not wanted
%\fax{312 996 1491}                          % optional, remove the line if not wanted
\email{vitor@vitorbaptista.com}                      % optional, remove the line if not wanted
\extrainfo{http://vitorbaptista.com} % optional, remove the line if not wanted
%\photo[64pt]{my_photo_smaller}                         % '64pt' is the height the picture must be resized to and 'picture' is the name of the picture file; optional, remove the line if not wanted
%\quote{"Be the change you want to see in the world."\newline Mahatma Gandhi}                 % optional, remove the line if not wanted

%\nopagenumbers{}                             % uncomment to suppress automatic page numbering for CVs longer than one page


%----------------------------------------------------------------------------------
%            content
%----------------------------------------------------------------------------------
\begin{document}
\maketitle

%\section{Master thesis}
%\cvline{title}{\emph{Title}}
%\cvline{supervisors}{Supervisors}
%\cvline{description}{\small Short thesis abstract}

\section{Experience}
%\subsection{Vocational}
\cventry{2012--2014}{Developer}{Open Knowledge Foundation}{Cambridge}{UK}
{
  I was part of the CKAN (\url{http://ckan.org}) team. CKAN is the world's most
  used open source data portal portal, powering sites like
  \url{http://data.gov.uk} and \url{http://data.gov}. It's written in Python
  using Pylons and Postgres. The core team was composed of about 10 people
  geographically distributed.
  \\
  My latest contribution was revamping its data
  visualization system, allowing a single dataset to have multiple
  visualizations, and making it easier to build custom visualizations. This
  feature was initially developed for London's Natural History Museum Data
  Portal (\url{http://data.nhm.ac.uk/}), and was later released as part of CKAN
  2.4.
  \\
  I've also deployed a few data portals like \url{http://datos.gob.mx} and
  \url{http://data.org.pk}, making sure their servers are secured and able to
  handle their expected load. customizing them based on the clients'
  requirements.
}
\cventry{2011--2012}{Developer}{ThoughtWorks}{Porto Alegre}{Brazil}
{
  I've worked as a Rails developer for a few projects during my time at
  ThoughtWorks. The projects were managed following agile development
  practices, including pair programming.
  \\
  My main contribution was being part of the team of 6 people that flew from
  Brazil to the USA to work on-site for 3 months with one of the (at the time)
  largest clients of ThoughtWorks. The objective was to meet the client and
  their internal developers, work with them, understand the code structure and
  their challenges, to later bring the project to Brazil.
  \\
  It was one of the largest and oldest Rails codebases worldwide, with about
  300k lines of code, in production since 2007. The team had 50+ people
  distributed between two locations in the USA, Canada, India, and Brazil.
}
\cventry{2010}{Intern}{Linux Foundation}{}{}
{
  I was selected as a student in Google's Summer of Code 2010 for the Linux Foundation.
  I've worked for 3 months with the leader of the OpenPrinting project, Till Kamppeter,
  developing in Python a compressor for PostScript Description Files (printer
  drivers) named pyppd (\url{https://github.com/vitorbaptista/pyppd}).  I was
  able to get 90\% of compression ratio. It's included by default in all the
  main GNU/Linux distributions, like Ubuntu, Debian and Fedora.
}
\cventry{2007--2010}{Researcher}{Digital Video Applications Lab}{João Pessoa}{Brazil}
{
  I've worked on the development of OpenGinga, a free implementation of the brazilian
  digital tv middleware, Ginga. It was done in C++ and Java and ran on a GNU/Linux box.
}

\section{Education}
\cventry{2013--2015}{Master's in Computer Science}
{Federal University of Paraíba (UFPB)}{Paraíba}{Brazil}
{
  I've built a statistical model using R and Python that detects when a
  legislator is changing her position in relation to the government, entering
  or leaving its coalition, based on her voting patterns during rollcalls. The
  final model was based on the C5.0 method. It achieved 90\% accuracy with an
  area under the ROC curve of 0.88.
}
\cventry{2006--2010}{Bachelor in Computer Science}
{Federal University of Paraíba (UFPB)}{Paraíba}{Brazil}
{
  I've built a software that returns if two or more software licenses are
  compatible or not. They have to be described in the Creative Commons Rights
  Expression Language (ccREL). This could help companies and developers to
  better understand if they can add in their projects a third-party library
  that's licensed under some other license.
}

\section{Projects}
\cvline{Orçamento ao seu alcance}
{\small 
  It was made by me with a designer and a project manager for a brazilian
  NGO that monitors the government spending. The main objective was to raise
  awareness of the problem of underspending by the Brazilian federal
  government. It was built in 6 weeks, from inception to production, as a
  single-page app using Angular 1.0, Rails 4.0, and NVD3.js.
  (\url{http://orcamento.inesc.org.br})
}
\cvline{Escola Que Queremos}
{\small 
  In Brazil, schools are ranked based on IDEB, the Basic Education's
  Development Index. It defines what's a good school based on approval rates
  and scores in Math and Portuguese. That's quite limiting: what about all the
  other things that are important as well, like offering sports, having a
  computer lab, or even basic things like offering lunch for the students.
  What's important to me might be irrelevant to you. So, during a 2-day
  hackathon, I built a tool with other 3 friends (2 devs and 1 journalist) that
  allows the user to build her own score, picking whatever matters to her. She
  can then compare the Brazilian schools using this custom score. I was
  responsible for the design and front-end programming. It was built as a
  single-page app using Rails 3.2 and D3.js. We won the first place.
  (\url{http://escolaquequeremos.org})
}
\cvline{Reputação S/A}
{\small 
  It's a mobile visualization and reference tool of the consumers
  complaints against Brazilian companies using data collected by PROCON
  (consumer protection organization) and made for the first contest organized
  by the Brazilian Ministry of Justice together the W3C office in Brazil. With
  a total project time of one week, from inception to production. We won the
  first place.
  (\url{http://reputacao-sa.org})
}
\cvline{Who won 2012's elections in Brazil?}
{\small
  I've worked with Estadão's data visualization team during the week
  before municipal/state elections in Brazil. We designed and built an
  interactive visualization that shows the growth and decrease of the political
  parties in the states. How many mayors they got elected, how many people
  they'll govern, etc..
  It's a single-page app built with D3.js. We updated the data live, while the
  results were being published. There were almost 150.000 visits in two days.
  (\url{http://estadaodados.com/eleicoes2012/})
}
\cvline{Retrato da Violência}
{\small
  I worked with a couple friends an interactive visualization on the data about
  rapes in the Brazilian state Rio Grande do Sul. This was for the DecodersRS
  contest organized by W3C Brazil. This was done in a couple weeks during our
  spare time. We won the first place.
  (\url{http://retratodaviolencia.org})
}

\vspace{.4in}\section{Skills}
\cvcomputer{Languages}{Ruby, Python, JavaScript, R, Bash} {}{}
\cvcomputer{Frameworks}{Rails, D3.js, NodeJS} {}{}

\end{document}


%% end of file `template_en.tex'.
