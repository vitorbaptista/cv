%% start of file `template_en.tex'.
%% Copyright 2007 Xavier Danaux (xdanaux@gmail.com).
%
% This work may be distributed and/or modified under the
% conditions of the LaTeX Project Public License version 1.3c,
% available at http://www.latex-project.org/lppl/.


\documentclass[11pt,a4paper]{moderncv}

% moderncv themes
%\moderncvtheme[blue]{casual}                 % optional argument are 'blue' (default), 'orange', 'red', 'green', 'grey' and 'roman' (for roman fonts, instead of sans serif fonts)
\moderncvtheme[green]{classic}                % idem

% character encoding
\usepackage[utf8]{inputenc}                   % replace by the encoding you are using

% adjust the page margins
\usepackage[scale=0.8]{geometry}
\recomputelengths                             % required when changes are made to page layout lengths

% personal data
\firstname{Vítor}
\familyname{Baptista}
\title{Web Developer}               % optional, remove the line if not wanted
%\address{\hspace{-1in}851 S. Morgan (M/C 249)}{Chicago, IL 60607-7045}    % optional, remove the line if not wanted
\mobile{+55 (83) 9630-0657}                    % optional, remove the line if not wanted
%\phone{(312) 413-8265}                      % optional, remove the line if not wanted
%\fax{312 996 1491}                          % optional, remove the line if not wanted
\email{vitor@vitorbaptista.com}                      % optional, remove the line if not wanted
\extrainfo{http://vitorbaptista.com} % optional, remove the line if not wanted
%\photo[64pt]{my_photo_smaller}                         % '64pt' is the height the picture must be resized to and 'picture' is the name of the picture file; optional, remove the line if not wanted
\quote{"Be the change you want to see in the world."\newline Mahatma Gandhi}                 % optional, remove the line if not wanted

%\nopagenumbers{}                             % uncomment to suppress automatic page numbering for CVs longer than one page


%----------------------------------------------------------------------------------
%            content
%----------------------------------------------------------------------------------
\begin{document}
\maketitle

%\section{Master thesis}
%\cvline{title}{\emph{Title}}
%\cvline{supervisors}{Supervisors}
%\cvline{description}{\small Short thesis abstract}

\section{Experience}
%\subsection{Vocational}
\cventry{2012--Current}{Developer}{Open Knowledge Foundation}{}{}
{
  I work on CKAN (\url{http://ckan.org}), a free software Python web application
  for Open Data Portals. It has been used by many governments, including
  US (\url{http://data.gov}), UK (\url{http://data.gov.uk}), and Brazil
  (\url{http://dados.gov.br}).
  Before that, I was part of OpenSpending (\url{http://openspending.org}) team.
  It's a tool to help working with monetary data, providing some visualizations
  and an API.
}
\cventry{2011--2012}{Developer}{ThoughtWorks}{Porto Alegre}{RS}
{
  I've worked as a Rails software consultant for a couple projects during my time
  at ThoughtWorks.
  The first one was with a team of 7 people, helping a startup in San Francisco
  be able to meet a deadline. The tech stack was Rails + JavaScript and solr for
  full-text search. It ran for 1.5 months.
  The last one was with an online vehicle auctions' site. It's one of the largest
  Rails codebases worldwide, with almost 300k lines of code, and in production
  since 2007. The team had 50+ people, distributed between the US, Brazil, Canada
  and India. I was part of the 6-people team that started developing it from
  Brazil.
}
\cventry{2007--2010}{Researcher}{Digital Video Applications Lab}{João Pessoa}{PB}
{
  I've worked on the development of OpenGinga, a free implementation of the brazilian
  digital tv middleware, Ginga. It was done in C++ and Java to run on a GNU/Linux box.
}
\cventry{2010}{Intern}{Linux Foundation}{}{}
{
  I was selected as a student in Google's Summer of Code 2010 for the Linux Foundation.
  I've worked for 3 months with the leader of the OpenPrinting project, Till Kamppeter,
  developing in Python a compressor for PostScript Description Files (printer drivers).
  I was able to get 90\% of compression ratio. It's included by default in all the main
  GNU/Linux distributions, like Ubuntu, Debian and Fedora.
}
%\cventry{1997--2000}{Applications Developer}{Information Resources Inc.}{Chicago}{IL}{Design, coding, and testing of market research applications.}                % arguments 3 to 6 are optional
%\cventry{1993-1996}{Statistical Analyst}{Kemper Insurance Companies}{Long Grove}{IL}{Statistical report filing to state agencies and the automation these reports.}                % arguments 3 to 6 are optional
%\cventry{Autumn 1991}{Teaching Assistant}{University of Illinois}{Chicago}{IL}{Lead the discussion sections for a large lecture course.}                % arguments 3 to 6 are optional
%\subsection{Miscellaneous}
%\cventry{year--year}{Job title}{Employer}{City}{}{Description line 1\newline{}Description line 2}% arguments 3 to 6 are optional

\section{Education}
\cventry{2013--2015 (expected)}{Master's in Computer Science}
{Federal University of Paraíba, Department of Informatics}{João Pessoa}{PB}
{
  Building a tool to increase the visibility into the law-making process by
  applying DVCSs to laws
}
\cventry{2006--2010}{Bachelor in Computer Science}
{Federal University of Paraíba, Department of Informatics}{João Pessoa}{PB}
{
  Bachelor thesis entitled ``A tool for the verification of compatibility between software licenses.''
}

\section{Projects}
\cvline{Orçamento ao seu alcance}
{\small 
  This is a visualization of the brazilian federal budget. We show how much of
  it went to each public body and, more importantly, how much was actually
  spent. We focus on underspending. For example, a person could go and check
  the Ministry of Education budget in 2012, and she might be surprised to learn
  that more than 16\% of its budget wasn't spent. Hopefully, this would raise
  questions, and she'll try to find out with her representatives what's
  happening.
  (\url{http://orcamento.inesc.org.br})
}
\cvline{Escola Que Queremos}
{\small 
  In Brazil, schools are ranked based on IDEB, the Basic Education's
  Development Index. It defines what's a good school, and it's used to plan
  where money will be invested.
  The index fails to show many nuances on what makes a good school, as it
  considers only approval rate and scores in Math and Portuguese. So we built a
  tool that allows you to build your own score, picking whatever matters to
  you, and compare other schools related to it.
  It was built in a 2-day hackathon. I did the design and front-end programming.
  (\url{http://escolaquequeremos.org})
}
\cvline{Reputação S/A}
{\small 
  It's a mobile visualization and reference tool of the consumers
  complaints against Brazilian companies using data collected by PROCON
  (consumer protection organization) and made for the first contest organized
  by the Brazilian Ministry of Justice together the W3C office in Brazil. With
  a total project time of one week, from inception to production. We won the
  first place.
  (\url{http://reputacao-sa.org})
}
\cvline{Who won 2012's elections in Brazil?}
{\small
  I've worked with Estadão's data visualization team during the week
  before municipal/state elections in Brazil. We designed and built an
  interactive visualization that shows the growth and decrease of the political
  parties in the states. How many mayors they got elected, how many people
  they'll govern, etc..
  It's a single-page app built with d3.js. We updated its data while the
  results were published, live. There were almost 150.000 visits in two days.
  (\url{http://estadaodados.com/eleicoes2012/})
}
\cvline{Retrato da Violência}
{\small
  I did with some friends an interactive visualization on the data about
  rapes in Rio Grande do Sul, a state in Brazil. We've worked on it for a few
  weeks from home, in our spare time, as our entry in the W3C's contest
  DecodersRS. We had lots of fun, and won the first place in the end.
  (\url{http://retratodaviolencia.org})
}
\cvline{pyppd}
{\small
  This is a compressor for PPD files, that are used somewhat like
  printer's drivers. It was useful for Ubuntu, because when deploying a new
  version, they had to fit everything in a CD. So, they had either to cut support
  for older printers, or leave other improvements out.
  With this project, I got >90\% compression rate. It freed them more than 20 MB,
  which is quite a lot in a 700 MB CD. Since then, it's being used by default in
  many other GNU/Linux distributions.
  (\url{http://github.com/vitorbaptista/pyppd/})
}

\vspace{.4in}\section{Skills}
\cvcomputer{Languages}{Ruby, JavaScript, Python, Bash, C, PHP, Java} {}{}
\cvcomputer{Frameworks}{Rails, Sinatra, Angular.js, D3.js, NodeJS, Raphaël.js} {}{}
%\cvcomputer{}{} {}{}

\clearpage

\section{Languages}
\cvlanguage{Portuguese}{Native}{}
\cvlanguage{English}{Advanced}{}
%\cvlanguage{language 3}{Skill level}{Comment}

\section{Other Interests}
\cvline{Free Software}{\small As a member of my state's GNU/Linux User Group,
I've helped organize the ENSOL in its 2008, 2009 and 2010 editions. It was one
of the biggest FLOSS meetings of Brazil's Northeast, attracting around 800
participants per year.}

\closesection{}                   % needed to renewcommands
\renewcommand{\listitemsymbol}{-} % change the symbol for lists

%\section{Extra 1}
%\cvlistitem{Item 1}
%\cvlistitem{Item 2}
%\cvlistitem[+]{Item 3}            % optional other symbol

%\section{Extra 2}
%\cvlistdoubleitem[\Neutral]{Item 1}{Item 4}
%\cvlistdoubleitem[\Neutral]{Item 2}{Item 5}
%cvlistdoubleitem[\Neutral]{Item 3}{}

% Publications from a BibTeX file
%\nocite{*}
%\bibliographystyle{plain}
%\bibliography{publications}       % 'publications' is the name of a BibTeX file

\end{document}


%% end of file `template_en.tex'.
