%% start of file `template_en.tex'.
%% Copyright 2007 Xavier Danaux (xdanaux@gmail.com).
%
% This work may be distributed and/or modified under the
% conditions of the LaTeX Project Public License version 1.3c,
% available at http://www.latex-project.org/lppl/.


\documentclass[11pt,a4paper]{moderncv}

% moderncv themes
%\moderncvtheme[blue]{casual}                 % optional argument are 'blue' (default), 'orange', 'red', 'green', 'grey' and 'roman' (for roman fonts, instead of sans serif fonts)
\moderncvtheme[green]{classic}                % idem

% character encoding
\usepackage[utf8]{inputenc}                   % replace by the encoding you are using

% adjust the page margins
\usepackage[scale=0.8]{geometry}
\recomputelengths                             % required when changes are made to page layout lengths

% personal data
\firstname{Vítor}
\familyname{Baptista}
\title{Cientista da Computação}               % optional, remove the line if not wanted
%\address{\hspace{-1in}851 S. Morgan (M/C 249)}{Chicago, IL 60607-7045}    % optional, remove the line if not wanted
\mobile{+55 (51) 8269-0229}                    % optional, remove the line if not wanted
%\phone{(312) 413-8265}                      % optional, remove the line if not wanted
%\fax{312 996 1491}                          % optional, remove the line if not wanted
\email{vitor@vitorbaptista.com}                      % optional, remove the line if not wanted
\extrainfo{http://vitorbaptista.com} % optional, remove the line if not wanted
%\photo[64pt]{my_photo_smaller}                         % '64pt' is the height the picture must be resized to and 'picture' is the name of the picture file; optional, remove the line if not wanted
\quote{"Seja a mudança que desejas ver no mundo."\newline Mahatma Gandhi}                 % optional, remove the line if not wanted

%\nopagenumbers{}                             % uncomment to suppress automatic page numbering for CVs longer than one page


%----------------------------------------------------------------------------------
%            content
%----------------------------------------------------------------------------------
\begin{document}
\maketitle

%\section{Master thesis}
%\cvline{title}{\emph{Title}}
%\cvline{supervisors}{Supervisors}
%\cvline{description}{\small Short thesis abstract}

\section{Experiência}
%\subsection{Vocational}
\cventry{2011--Atual}{Desenvolvedor}{ThoughtWorks}{Porto Alegre}{RS}
{Trabalho no desenvolvimento de um site americano de leilão de carros.
É um sistema com mais de 5 anos e 250 mil linhas de código, e cerca de 50
pessoas na equipe, distribuídas entre Brasil, Chicago, Atlanta e Índia.
Usamos processos de desenvolvimento ágil, com reuniões diárias por vídeo, testes de software,
programação em par, etc.}
\cventry{2007--2010}{Pesquisador}{Laboratório de Aplicações de Vídeo
Digital}{João Pessoa}{PB}{Trabalhei no projeto de desenvolvimento do
OpenGinga, uma implementação livre do middleware brasileiro de TV Digital,
o Ginga. Feito em C++ e Java para o GNU/Linux.}                % arguments 3 to 6 are optional
\cventry{2010}{Estagiário}{Linux Foundation}{}{}{Fui selecionado como um
estudante no Google's Summer of Code 2010 para a Linux Foundation. Trabalhei
por 3 meses com o líder do projeto OpenPrinting, Till Kamppeter,
desenvolvendo um compressor para PostScript Description Files (drivers de
impressoras). Ele foi feito em Python e consegui uma taxa de compressão
superior a 90\%. É incluso por padrão em diversas distribuições GNU/Linux,
inclusive no Ubuntu.}% arguments 3 to 6 are optional
%\cventry{1997--2000}{Applications Developer}{Information Resources Inc.}{Chicago}{IL}{Design, coding, and testing of market research applications.}                % arguments 3 to 6 are optional
%\cventry{1993-1996}{Statistical Analyst}{Kemper Insurance Companies}{Long Grove}{IL}{Statistical report filing to state agencies and the automation these reports.}                % arguments 3 to 6 are optional
%\cventry{Autumn 1991}{Teaching Assistant}{University of Illinois}{Chicago}{IL}{Lead the discussion sections for a large lecture course.}                % arguments 3 to 6 are optional
%\subsection{Miscellaneous}
%\cventry{year--year}{Job title}{Employer}{City}{}{Description line 1\newline{}Description line 2}% arguments 3 to 6 are optional

\section{Educação}
\cventry{2006--2010}{Bachalerando de Ciência da Computação}{Universidade Federal da Paraíba, Departamento de Informática}{João Pessoa}{PB}{Monografia entitulada ``Uma ferramenta para verificação de compatibilidade entre licenças de software.''}

\vspace{.4in}\section{Habilidades}
\cvcomputer{Linguagens}{Ruby, JavaScript, Bash, Python, C, PHP, Java} {}{}
\cvcomputer{Frameworks} {Rails, Sinatra, Backbone.js} {}{}
%\cvcomputer{}{} {}{}

\section{Línguas}
\cvlanguage{Português}{Nativo}{}
\cvlanguage{Inglês}{Avançado}{}
%\cvlanguage{language 3}{Skill level}{Comment}

\section{Projetos Pessoais}
\cvline{WeLoveIran.org}{\small Projeto feito com alguns amigos para ajudar a campanha
do IsraelLovesIran.com, um site onde pessoas de ambos países podem trocar depoimentos
e mensagens de amor, para mostrar que eles não apóiam uma possível guerra
(\url{https://github.com/rafaelpetry/israelovesiran}) }
\cvline{Transparência Pública}{\small Pequenos projetos relacionados a dados abertos
e transparência pública, principalmente tentando entender melhor a realidade paraibana
(\url{http://vitorbaptista.com/category/transparencia/})}

\section{Interesses}
\cvline{Software Livre}{\small Como membro do Grupo de Usuários GNU/Linux
da Paraíba, ajudei a organizar diversos eventos de Software Livre na
região, inclusive o ENSOL, um dos maiores do Brasil.}

\closesection{}                   % needed to renewcommands
\renewcommand{\listitemsymbol}{-} % change the symbol for lists

%\section{Extra 1}
%\cvlistitem{Item 1}
%\cvlistitem{Item 2}
%\cvlistitem[+]{Item 3}            % optional other symbol

%\section{Extra 2}
%\cvlistdoubleitem[\Neutral]{Item 1}{Item 4}
%\cvlistdoubleitem[\Neutral]{Item 2}{Item 5}
%cvlistdoubleitem[\Neutral]{Item 3}{}

% Publications from a BibTeX file
%\nocite{*}
%\bibliographystyle{plain}
%\bibliography{publications}       % 'publications' is the name of a BibTeX file

\end{document}


%% end of file `template_en.tex'.
